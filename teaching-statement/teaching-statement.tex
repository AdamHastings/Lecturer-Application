\documentclass[11pt]{amsart}

% Set page margins
\usepackage[margin=1in,top=0.3in]{geometry}

% Mathematical things
\setlength{\parskip}{2pt plus 1pt minus 1pt}

\begin{document}
\title{Teaching statement}
\author{Christian Schnell}
\maketitle
\thispagestyle{empty}


Mathematics is unique as a subject, in that it comprises both highly abstract theory
and very concrete applications to practical problems. Teaching mathematics, no matter
at what level, has to reflect this duality: in an introductory calculus course, for
instance, students need to master derivatives and integrals, but also need to learn
how to apply calculus to real-world problems; in a graduate course, students need to
become familiar with a new area and its results, but also need to understand what
those results mean and how to think about them concretely.

In my teaching, I always try to arrange the subject matter in such a way that my
listeners can understand and appreciate both aspects. I show many examples to
illustrate the meaning of a new concept, to justify why a certain theorem must be
true, or to suggest a different point of view on a result. For instance,
when introducing the product rule for taking derivatives, I give the proof; but in
addition, I show students how the area of a rectangle changes with its sides, and how
the product rule can be understood from the resulting simple formula $\Delta(uv)
\approx (\Delta u)v + u (\Delta v)$. When preparing, be it for a lecture or for a
research talk, I spend a lot of time on finding the correct order of presentation, in
which each step is well motivated. I also strive for clarity in my exposition by
being as nontechnical as possible. It has happened several times that students from
unrelated courses began attending my class on the side, because they liked my way of
explaining things; for instance, there are again two such students in my course on abstract
algebra this semester.

Of course, I know that my teaching is not perfect, and so I try to improve by
taking hints from other good teachers. For my courses at UIC, I allow students to
submit anonymous comments or suggestions through my web page, and I act on these
suggestions where possible. In my first year of graduate school, I also attended a
two-month teacher training program, which included discussions about teaching, and
practice lectures that were videotaped and then analyzed. 

Over the past eight years, I have been teaching many different types of courses.  In
graduate school, I have been both teaching assistant and instructor for
single-variable and multi-variable calculus, differential equations, and for a course
about actuarial mathematics. I helped to teach a class for elementary school
teachers, and during two quarters, I was the teaching assistant in a graduate course
on algebraic geometry. All of those courses had no more than 30 students. At UIC, I
have taught single-variable calculus to larger groups of 70 to 80 students; I have
also given an advanced course on complex manifolds to graduate students, and I am
currently teaching a class on abstract algebra to mathematics majors. At a much more
advanced level, I have also given lectures and prepared exercise sessions at two
summer schools on Hodge theory.

I realize that, while mathematics is often taught to a fairly large number of
students at once, learning or understanding something is an individual act. As a
graduate student, the most successful course that I helped to teach was my adviser's
class on mathematical concepts for elementary school teachers: instead of lectures,
the course relied entirely on worksheets, pair work, and student presentations, and
as a result, every one of the prospective teachers mastered the mathematical content
and became good at explaining it to others. Since my days as a teaching assistant, I
enjoy answering student questions individually, and helping a single person
understand something. For example, there was a student in my calculus course last
year who, although very intelligent, often became confused after reading the
textbook. He also had a poor score on his midterm exam, and was worried that he
might not make it into pharmacy school. He started coming to my office hours
regularly; each time, I would carefully listen to what he did not understand, and
then clear up his confusion.  In the end, he had a very good final exam, I was able
to write him a letter of recommendation, and he is now happily studying pharmacy.
This example shows that individual attention is very important. I therefore try to be
very accessible to students, be it during class, during office hours, or through
answering questions by email. 



\end{document}
