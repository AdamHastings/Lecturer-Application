\documentclass[11pt]{amsart}

% Set page margins
\usepackage[margin=1in,top=1in]{geometry}

% Mathematical things
\setlength{\parskip}{2pt plus 1pt minus 1pt}

\begin{document}
\title{Teaching statement}
\author{Adam Hastings}
\maketitle
% \thispagestyle{empty}


%%% philosophy
\textbf{Teaching Philosophy:}
My teaching philosophy is centered on my own experiences as a student and lifelong learner. 

Perhaps surprisingly, I am also a big believer in teaching ``wrong'' approaches: Computer science is a mix of engineering tradeoffs and legacy decisions, and understanding \textit{why} things are the way they are is often a ---- 

First and foremost, this means that 

Build up students' confidence 

I recently read Feynman's memoir (Brazil education by rote memorization)

I have never once identified with the ``genius''-style student who 

I come from a long line of teachers (but none at the university level).

Learning should be fun. But not entertainment. 

%%% experience
\textbf{Teaching Experience:} 
My first job was teaching: 
When I was 16, I taught four students aged 8 to 12 the basics of piano technique and performance. 
For two years I developed weekly lesson plans, exercises, ``homework'' (i.e. practice assignments), and even final ``exams'' (performance recitals with the other students).
The fact that my students were (according to their parents) excited each week for lessons while making tremendous progress was as good of a measure of success as I could have possibly hoped for.
%  (seeing a student go from zero to playing Beethoven in 18 months is exhilarating),
% This experience developed my ability to work one-on-one with students, motivate and encourage students who are frustrated or struggling, and practice the . 

% Teaching 
My interest and dedication to teaching has persisted since then, and was one of the primary reasons I pursued a PhD.
During the Spring '24 semester, I was fortunate enough to be able to --- by d

developed and taught a new course offering at Columbia as a Teaching Fellow under the supervision of my advisor, Prof. Simha Sethumadhavan.
This class---The Economics of Cybersecurity (COMS~6998:sec12)---was a seminar-style class where students read and discussed papers on the economic issues that underlie security  as well as economic approaches towards addressing them.
Given the readiness level of the class (comprised of Masters students and senior undergraduates) and the novelty of the economics concepts, I found it necessary to also dedicate about half of class time to lectures. 
This experience taught me how to craft a course syllabus and reading list, how to create assignments and quizzes, how to give a two-hour lecture, how to manage a classroom, and how to take myself as someone who can teach exceptional students at the collegiate level. 

% TAing
In addition to teaching as a lecturer, I have had numerous experiences teaching students as a TA/CA:
During my time at BYU, I TA'd 
Elements of Electrical Engineering, Embedded Systems, Junior Team Design Project (a semester-long group project involving signal processing, circuit design, and embedded systems), and Advanced Embedded Linux Systems (twice).
At Columbia I have been a Course Assistant for Computer Architecture (three times, including under two professors), Security I, and Hardware Security.
In these roles I worked with students of all backgrounds and experience levels. 
%%% mentoring
In addition to TA work, I have mentored three undergraduate students as research interns. 

% In all experiences, I have 

%%% classes
\textbf{Teaching Interests:}
I am most interested in teaching classes in the computer engineering and security tracks: Fundamentals of Computer Systems (CSEE 3827), Advanced Programming (COMS 3157), Data Structures (COMS 3134), Computer Architecture (CSEE 4824), and Embedded Systems (COMS 4840), and Hardware Security (COMS 6242). I also believe I would excel at teaching entry-level computer science classes including Intro to CS (COMS 1004), Intro to Computing for Engineers and Applied Scientists (COMS 1006), and Discrete Mathematics (COMS 3203). 
Finally, I would be delighted to continue teaching my graduate-level seminar The Economics of Cybersecurity (COMS 6998:sec12). 


% Developing the security curriculum
I am interested in more than teaching existing classes. 
It has long been my opinion that the Columbia Department of Computer Science would greatly benefit from a more hands-on security education. 
If selected for this role, one of my goals would be to develop a learning-by-doing curriculum for the security track. 

%%% conclusion

\end{document}
